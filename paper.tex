\documentclass[aps,twocolumn,superscriptaddress,showpacs,showkeys]{revtex4-1}
%\documentclass[aps,superscriptaddress,showpacs]{revtex4}
\usepackage{graphics,graphicx,dcolumn,bm,fleqn,epic,eepic,float,epsfig}
\usepackage{amssymb,amsmath,multirow,rotate,color,float}
\usepackage{epstopdf}
\usepackage{times}
\usepackage{color}
\usepackage{soul}                    % package needed for overstriking
\definecolor{red}{rgb}{1,0,0}
\definecolor{green}{rgb}{0,1,0}
\definecolor{blue}{rgb}{0,0,1}
\newcommand{\xxr}[1]{\textcolor{red}{#1}}
\newcommand{\xxb}[1]{\textcolor{blue}{#1}}
%\usepackage{epsfig}
%\usepackage{graphicx}
\newcommand{\erf}{\text{erf}}

%%%%%%%%%%%%%%% macro %%%%%%%%%%%%%%%
\newcommand{\lc}{{l_{\mathrm{c}}}}  %characteristic lenght
\newcommand{\dpl}{\nabla p}         %pressure gradient   
\newcommand{\q}{\vbf{q}}            %volumetric flow
\newcommand{\area}{A}               %cross-section area

\newcommand{\BLA}{BLA BLA BLA BLA BLA BLA BLA BLA BLA BLA BLA BLA BLA BLA BLA }

\newcommand{\fracd}[2]{
\displaystyle{
\frac{ \displaystyle{#1} }{ \displaystyle{#2} }
}
}

\newcommand{\diff}[1]{\,{\rm d}{#1}}
\newcommand{\diag}{{\rm diag}}

\newcommand{\taub}{\tau_{\rm bulk}}

\newcommand{\ii}{{\dot{\imath}}}
\newcommand{\jj}{{\dot{\jmath}}}
\newcommand{\ib}{\hat{\imath}}
\renewcommand{\c}{{\bf c}}
\newcommand{\ci}{\c_{\ii}}

\newcommand{\vbf}[1]{\mathbf{#1}}
\newcommand{\dt}{\Delta t} 
\newcommand{\dx}{\Delta x}
\newcommand{\rhor}{{\rho\!_{_{\circ}}}}
\newcommand{\drho}{\delta\!\rho}

\newcommand{\acc}{g}
\newcommand{\bfacc}{\vbf{\acc}}

\newcommand{\dc}{d_{\rm c}}

\newcommand{\uvel}{\vbf{u}}
\newcommand{\ui}{u_{\ai}}
\newcommand{\uj}{u_{\aj}}
\newcommand{\uk}{u_{\ak}}
\newcommand{\us}{{u_{\rm s}}}

\newcommand{\vvel}{\vbf{v}}
\newcommand{\vi}{v_{\ai}}
\newcommand{\vj}{v_{\aj}}
\newcommand{\vk}{v_{\ak}}

\newcommand{\x}{\vbf{x}}
\renewcommand{\xi}{x_{\ai}}
\newcommand{\xj}{x_{\aj}}
\newcommand{\xk}{x_{\ak}}

\renewcommand{\Xi}{X_{\ai}}
\newcommand{\Xj}{X_{\aj}}
\newcommand{\Xk}{X_{\ak}}
\newcommand{\Xs}{X_{\mc{S}}}
\newcommand{\Xc}{X_{\mc{C}}}

\newcommand{\ai}{1}
\newcommand{\aj}{2}
\newcommand{\ak}{3}

\newcommand{\e}{\vbf{e}}
\newcommand{\ei}{\e_{\ai}}
\newcommand{\ej}{\e_{\aj}}
\newcommand{\ek}{\e_{\ak}}

\newcommand{\f}{f}
\newcommand{\bff}{\vbf{\f}}
\newcommand{\bffq}{{\bff}^{\rm eq }}
\newcommand{\fii}{\f_{\ii}}
\newcommand{\fq}{\f^{\rm eq}}
\newcommand{\fqi}{\fq_{\ii}}
\newcommand{\fj}{\f_{\jj}}
\newcommand{\fqj}{\fq_{\jj}}

\newcommand{\Vf}{V_{\rm f}}
\newcommand{\Vt}{V_{\rm t}}

\newcommand{\Ti}{T_{\rm i}}
\newcommand{\Tf}{T_{\rm f}}

\newcommand{\bfOmega}{\boldsymbol{\Omega}}
\newcommand{\wwi}{\omega_{\ii}}
\newcommand{\cs}{{c_{\rm s}}}

\newcommand{\crcl}{\text{\tiny{$\bigcirc$}}}

\newcommand{\FEM}{\ensuremath{\text{FEM}}}
\newcommand{\LB}{\ensuremath{\text{LB}}}
\newcommand{\BGK}{\ensuremath{\text{BGK}}}
\newcommand{\LBBGK}{\ensuremath{\text{LB-BGK}}}
\newcommand{\MRT}{\ensuremath{\text{MRT}}}
\newcommand{\LBMRT}{\ensuremath{\text{LB-MRT}}}
\newcommand{\TRT}{\ensuremath{\text{TRT}}}

\newcommand{\lr}{\ensuremath{\text{l.r.}}}
\newcommand{\ir}{\ensuremath{\text{i.r.}}}
\newcommand{\hr}{\ensuremath{\text{h.r.}}}

\newcommand{\CFD}{\ensuremath{\text{CFD}}}
\newcommand{\mD}{\rm mD}

\renewcommand{\Re}{\ensuremath{\text{Re}}}
\newcommand{\Ma}{\ensuremath{\rm Ma}}
\newcommand{\Kn}{\ensuremath{\rm Kn}}

\newcommand{\mc}[1]{\mathcal{#1}}
\newcommand{\etal}{\sl et~al.}
\newcommand{\pwr}[1]{\!\times\!10\sp{#1}}
\newcommand{\pwrr}[1]{10\sp{#1}}
\newcommand{\avrg}[1]{\langle #1 \rangle}
\newcommand{\avrgu}{\avrg{u}}

\newcommand{\m}{m}
\newcommand{\mi}{\m_{\ii}}
\newcommand{\bfm}{\vbf{\m}}
\newcommand{\mq}{\m^{\mathrm{eq}}}
\newcommand{\mqi}{\m^{\mathrm{eq}}_{\ii}}
\newcommand{\bfmq}{{\bfm}^{\mathrm{eq}}}

\newcommand{\bfM}{\vbf{M}}
\newcommand{\bfS}{\vbf{S}}
\newcommand{\bfhS}{\vbf{\check{S}}}
\newcommand{\hS}{\check{S}}
%%%%%%%%%%%%%%% macro %%%%%%%%%%%%%%%

\begin{document}

\title{Transport properties of deposition of ellipsoidal particles}

\author{Bibhu Biswal}
\email{bbiswal@svc.ac.in}
\affiliation{New Delhi}

\author{Reza M.~Baram}
\email{reza@cii.fc.ul.pt}
\affiliation{Center for Theoretical and Computational Physics, 
             University of Lisbon,
             Av.~Prof.~Gama Pinto 2, 1649-003 Lisboa, Portugal}

\author{Jens Harting}
\email{j.harting@tue.nl}
\affiliation{ Department of Applied
   Physics, TU Eindhoven, P.O. Box 513, NL-5600MB Eindhoven, The 
   Netherlands }


\author{Pedro G.~Lind}
\email{plind@cii.fc.ul.pt}
\affiliation{Center for Theoretical and Computational Physics, 
             University of Lisbon,
             Av.~Prof.~Gama Pinto 2, 1649-003 Lisboa, Portugal}
\affiliation{Department of Physics, Faculty of Sciences of the
             University of Lisbon, 1649-003 Lisboa, Portugal} 

\author{Ariel Narv\'{a}ez}
\email{a.e.narvaez.salazar@tue.nl}
\affiliation{ Department of Applied
   Physics, TU Eindhoven, P.O. Box 513, NL-5600MB Eindhoven, The 
   Netherlands }

\date{\today}

\begin{abstract}
  We study the transport properties of depositions of ellipsoidal particles
  which has been generated using computer simulations. To be written.
\end{abstract}

%%%%PACS e Keywords
\pacs{
81.05.Rm,   %Granular materials, fabrication
45.70.-n,   %Granular systems, classical mechanics of
05.20.-y,   %Classical statistical mechanics
45.70.Cc    %Sandpile modelsk, Compaction granular systems
}

\keywords{Ellipsoidal, Deposition, Porous media, Permeability}

\maketitle

%%%%%%%%%%%%%%%%%%%%%%%%%%%%%%%%%%%%%%%%%%%%%%%%%%%%%%%%%%%%%%%%%%%%%%%%%%%%

\section{Introduction}
\label{sec:intro}
Predicting the transport properties of fluids in porous media based on their
micro-structure is a challenging problem which has been the focus of many
researches over last decades. For example, there have been many attempts to
relate the permeability to a relevant length scale characterizing the
geometrical structure of the porous medium.  This makes sense since the
permeability $\kappa$ has units of length squared, as defined by Darcy's
law~\cite{bib:darcy,bib:scheidegger}
%
\begin{equation}
\frac{\q}{\area} = -\frac{\kappa}{\mu} \dpl,
\end{equation}
%
where $\q$ is the volumetric flow, $\area$ is the cross-sectional area of the
sample, then $\q/\area$ is the flux, $\dpl$ is the pressure gradient across
the sample and $\mu$ is the fluid viscosity.  There are numerous approaches
based on some sort of length scales, such as hydraulic radius (the ratio of
volume to surface of the porous material), which fail to give satisfactory
predictions, except for simple cases.

Katz and Thompson~\cite{PhysRevB.34.8179} reasoned that these attempts
appeal to macroscopic geometrical properties and ignore the micro-structure of
the porous material.  They derived, using percolation theory, a relation for
the ratio of permeability $\kappa$ to electrical conductivity $\sigma$:
%
\begin{equation}
\frac{\kappa}{\sigma} = c\frac{\lc^2}{\sigma_0},
\label{eq:katz-thompson}
\end{equation}
%
where $\sigma_0$ is the electrical conductivity of the saturating fluid, $c$
is a constant of the order of $1/226$ and $\lc$ is a characteristic
length, which is estimated experimentally using mercury porosimetry
measurements performed on rock samples. The results obtained for permeability
using Eq.~\eqref{eq:katz-thompson} show impressive agreement with the
experimental measurements.

The characteristic length $\lc$ is obtained by finding the pathway which is
both percolating and consists of largest spheres which be inscribed in the
pore.

In this study, we perform a computational study on the permeability of
depositions of ellipsoid particles obtained in a recent work by two of the
authors~\cite{bib:reza-pedro2011}. We show that the permeability is strongly
affected by the anisotropy of the samples [up to 5 times]. Furthermore, We
will show that, while Katz-Thompson relation is valid when the permeability is
measured in $x$- $y$-direction, it fails completely in $z$-direction.
%
\begin{figure}
\includegraphics*[width=1.0\columnwidth]{fig1}
\caption{
\label{fig:deposit}
A deposit consisting of $N \sim 3000 $ ellipsoids with shape parameters
$\eta=2$ and $\zeta=1.5$.  Colors are arbitrarily chosen for better
visualization.  }
\end{figure}
%
 
The outline of the paper is as follows.  In Sec.~\ref{sec:simulations} we
describe preparation of the samples and the methods using which numerical
measurements are performed.  In Sec.~\ref{sec:permeability} we present the
results for permeability measurement.  We examine the validity of
Katz-Thompson relation for our samples in Sec.~\ref{sec:lc}.  Discussion and
conclusions are given in Sec.~\ref{sec:conclusions}.

%%%%%%%%%%%%%%%%%%%%%%%%%%%%%%%%%%%%%%%%%%%%%%%%%%%%%%%%%%%%%%%%%%%%%%%%
\section{Sample preparation and computational methods}
\label{sec:simulations}
For our study, we use deposits of ellipsoidal particles obtained using
Molecular Dynamics simulations as described in~\cite{bib:reza-pedro2011}. The
samples are generated by releasing such particles randomly inside a unit box,
depositing in $z$-direction under gravity. The box has periodic boundaries in
$x$- and $y$-directions, limited from below by a wall and open from above. The
particles are added to the system until the deposit attains a height slightly
larger than one. In order to reduce the boundary effect, in all our analyses a
layer of thickness $0.15\%$ from the bottom and $0.05\%$ from the top are
removed.

The shape of an ellipsoid is characterized by two parameters here defined as
$\eta = a/b > 1$ and $\zeta = b/c > 1$, with $a \ge b \ge c$ being the
semi-axis radii of the ellipsoid. Therefore, $\eta=1$ and $\zeta>1$
corresponds to an oblate and $\eta>1$ and $\zeta=1$ to a prolate, while
$\eta>1$ and $\zeta>1$ is for a general ellipsoid.

The size $r$ of an ellipsoid is defined as the radius of the sphere of equal
volume, defined as
%
\begin{equation}
\label{eq:radius}
  r = \sqrt[3]{a\,b\,c}.
\end{equation}
%
Therefore, the shape parameters $\eta \le 1$ and $\zeta \le 1$
together with the size $r>0$ fully specify an ellipsoid in a one-to-one
manner.

The surface of an ellipsoid can be approximately estimated by Knud Thomsen's
formula
%
\begin{equation}
\label{eq:surface}
 s = 4\pi\sqrt[p]{ \frac{ (a\,b)^{p} + (a\,c)^{p} + (b\,c)^{p} }{3} },
\end{equation}
%
where $p=1.6075$ yields a relative error of at most $1.061\%$. The sphericity
%
\begin{equation}
\label{eq:sphericity}
 \Phi = \frac{4\pi r}{s},
\end{equation}
%
is defined by the ratio of the surface area of a sphere with the same volume to
the surface area of the particle.
 
In this work, all the samples under the consideration consist of identical
particles. Therefore, we will refer to a sample by the parameters of the
compositing particle. Fig.~\ref{fig:deposit} shows a deposit consisting of
$N \sim 4200$ ellipsoids with $\eta=2$, $\zeta=1.5$, and $r=0.04$.
%
\begin{figure}
\includegraphics*[width=1.0\columnwidth]{data-figs/porosity}
\caption{
\label{fig:porosity}
Porosity of deposits as a function of the shape. $\eta$ and $\zeta$ are the
shape parameter of the compositing particles (see text.)
}
\end{figure}
%

To study the effect of the shape on the properties of the deposits, samples
with the shape parameters from the range $1 \le \eta$ and $\zeta \le 2.5$ are
considered. It is known that the shape of the particles a strong effect on
porosity of the deposits and, as it deviates from a sphere, leads to stronger
anisotropy in the contact network and particle
orientations~\cite{bib:reza-pedro2011}. Fig.~\ref{fig:porosity} shows the
porosity $\phi$ of deposits as a function of the shape.

%%%%%%%%%%%%%%%%%%%%%%%%%%%%%%%%%%%%%%%%%%%%%%%%%%%%%%%%%%%%%%%%%%%%%%%%%%
\section{Permeability}
\label{sec:permeability}
The calculation of permeability is done using the lattice Boltzmann
method. Many of its properties has been responsible for its popularity in this
field, i.e. discretization, easy implementation, and parallel running. The
integration of the Boltzmann equation on a regular lattice, defined by $\dx$,
with the discrete velocity restricted to the values $\ci$, in unit of
$\dx/\dt$, where $\dt$ represents the time step, provides the basic difference
equation for the lattice-Boltzmann method
%
\begin{equation}
\label{LB:MRT:eq}
\begin{split}
& \fii(\x+\dt\,\ci,t+\dt)  - \fii(\x,t) = \\
& \quad -\dt \left[{\bfM}^{-1} \cdot \bfhS \cdot \left( \bfm(\x,t) -
 \bfmq(\x,t) \right) \right]_{\ii},
\end{split}
\end{equation}
%
where $\fii(\x,t)$ represent the number of particles moving at the point $\x$
with velocity $\ci$ at time $t$. The right hand term represents the collision
operator presented by the multirelaxation time method
(\MRT)~\cite{2002RSPTA.360..437D} instead of by the most simple-popular
Bhatnagar, Gross and Krook (\BGK) ``single relation
time''~\cite{bib:chen-chen-martinez-matthaeus,bib:bgk}.  The \MRT\ collision
operator reduces well known drawback of the \BGK\ regarding the bounce-back
no-slip boundary conditions~\cite{bib:cf.CPaLLuCMi.2006,NZRHH10}. The $\bfM$
is a linear transformation chosen as such that the moments
%
\begin{equation}
\mi (\x,t)=\sum_{\jj} M_{\ii\,\jj} \, \f_{\jj}(\x,t)
\end{equation}
%       
represent hydrodynamic modes of the problem. We use the definitions given
in~\cite{2002RSPTA.360..437D}, where $\m_{1}$ is the fluid density, $\m_{2}$
represents the energy, $\mi$ with $\ii=4,6,8$ the momentum flux and $\mi$,
with $\ii=10,12,14,15,16$ are components of the symmetric traceless stress
tensor.  During the collision step the density and the momentum flux are
conserved so that $\mq_{1}=\m_{1}$ and $\mi=\mqi$ with $i=2,4,6$. The
non-conserved equilibrium moments $\mqi$, $i\neq 1,2,4,6$, are assumed to be
functions of these conserved moments and explicitly given
in~\cite{2002RSPTA.360..437D}. 
$\bfhS$ is a diagonal matrix $\hS_{\ii\,\jj} =
\check{s}_{\ii} \, \delta_{\ii\,\jj}$. The diagonal element
$\tau_{\ii}=1/\check{s}_{\ii}$ in the collision matrix is the relaxation time
of the moment $\mi$. One has
$\check{s}_{1}=\check{s}_{4}=\check{s}_{6}=\check{s}_{8}=0$, because the
corresponding moments are conserved. $\check{s}_{2}=1/\taub$ describes the
relaxation of the energy and
$\check{s}_{10}=\check{s}_{12}=\check{s}_{14}=\check{s}_{15}=\check{s}_{16}=1/
\tau$ the relaxation of the stress tensor components. The remaining diagonal
elements of $\bfhS$ are chosen as
%
\begin{equation}
\begin{split}
\bfhS = \diag & (0,1/\taub,1.4,0,1.2,0,1.2,0,1.2,1/\tau, \\
& 1.4,1/\tau,1.4,1/\tau,1/\tau,1/\tau,1.98,1.98,1.98),
\end{split}
\end{equation}
%
to optimize the algorithm
performance~\cite{2000PhRvE..61.6546L,2002RSPTA.360..437D}.  These two
relaxation times $\tau$ and $\taub$, restricted to be $>\dt/2$, remain free to
define the kinematic and bulk viscosity, respectively.

The macroscopic density $\rho(\x,t)$ and velocity $\uvel(\x,t)$ are obtained
from $\fii(\x,t)$ as
%
\begin{eqnarray}
\label{rho:eq}
\rho(\x,t) & = & \rhor \sum_{\ii} \fii(\x,t), \\
\label{uvel:eq}
\uvel(\x,t) & = &
\frac{\rhor}{\rho(\x,t)} \sum_{\ii}\fii(\x,t) \, \ci,
\end{eqnarray}
%
where $\rhor$ is a reference density. The pressure is given by
%
\begin{equation}
\label{eq:p}
p(\x,t) = {\cs}^{2} \, \rho(\x,t),
\end{equation}
%
where $\cs = 1/\sqrt{3}(\dx/\dt)$ is the speed of
sound~\cite{bib:qian-dhumieres-lallemand,bib:succi-01}. The kinematic
viscosity of the fluid is a function of the discretization parameters, $\dx$
and $\dt$, and the relaxation time
$\tau$~\cite{bib:chapman-cowling,Wolf05}. It is given by
%
\begin{equation}
\label{nu:eq}
\nu = \cs^2 \, \dt \left( \frac{\tau}{\dt} - \frac{1}{2} \right).
\end{equation}
%
The reference density is set to $\rhor = \pwrr{3}\,
\mathrm{kg}\,\mathrm{m}^{-3}$ and the relaxation times are kept at $\taub/\dt
= 0.80$ and $\taub/\dt = 0.84$.

On the solid walls, no-slip boundary conditions are implemented using within
the \LB\ method mid-plane bounce back collision
rules~\cite{SukopThorne2007}. To drive the fluid a pressure drop is impose
setting the density, see Eq.~\eqref{eq:p}, at the inlet and at the
outlet. This is done by on-site boundary conditions using the Zou \& He
method~\cite{bib:pf.QZoXHe.1997,HH08b}. Periodic boundary conditions are
implemented in the other directions.


The lattice Boltzmann simulations are performed on the discretized version of
the samples. Here, we use a 256~cubic cells per unit length in each
direction. This resolution has been chosen such that it minimizes the
computation time without loss of significant accuracy, see Fig.~\ref{fig:resolution}.
%
\begin{figure}
\includegraphics*[width=1.0\columnwidth]{data-figs/resolution}
\caption{
Error in the permeability estimation for diferent resolution.
The sample $\eta =? $ and $\zeta = ?$.
The realtive error is calculated comparing the results with the one obtained
using the 768~resolution sample. There is no much
\label{fig:resolution}
}
\end{figure}
%
%
\begin{figure}
\begin{tabular}{l l}
A) & B) \\
\includegraphics*[width=0.48\columnwidth]{data-figs/permeability_x}
&
\includegraphics*[width=0.48\columnwidth]{data-figs/permeability_z} \\
\end{tabular}

\flushleft C) 

\includegraphics*[width=1.0\columnwidth]{data-figs/permeability_ratio}
\caption{
Permeability as a function of the shape. $\eta$ and $\zeta$ 
are the shape parameter of the compositing particles (see text.).
Ratio of permeabilities in $z$- and $x$-directions as a function of the shape.
\label{fig:perm}
}
\end{figure}
%
As mentioned briefly in the previous section, the deposits are anisotropic in
$z$-direction.  Therefore, for all quantities under consideration we perform
measurement in two directions, vertical horizontal.

Fig.~\ref{fig:perm}~A) and~B) show the permeabilities measured in $x$- and
$z$-direction as functions of the shape. One can see that the permeability in
$z$-direction, $\kappa_z$, has generally a lower value than in $x$-direction,
$\kappa_x$. This difference becomes more pronounced as the shape deviates from
the sphere. This is better seen from the Fig.~\ref{fig:perm}~C) which shows
the ratio $\kappa_z/\kappa_x$.

This is due to the fact that the particles tend to lie in the horizontal
plane, increasing the surface area face the flow from vertical direction.
This is also the reason why flattening the particles (increasing $\zeta$) has
strong effect on the permeability while elongation (increasing $\eta$) has no
significant effect.
%
\begin{figure*}
\begin{tabular}{l l l l}
A) & B) & C) & D) \\
\includegraphics*[width=0.48\columnwidth]{data-figs/lc_x} &
\includegraphics*[width=0.48\columnwidth]{data-figs/lc_z} &
\includegraphics*[width=0.48\columnwidth]{data-figs/k_lc_x} &
\includegraphics*[width=0.48\columnwidth]{data-figs/k_lc_z}
\end{tabular}
\caption{
A)~Characteristic length $\lc$ in direction $x$.
B)~Characteristic length $\lc$ in direction $z$.
C)~Permeability normalized by the characteristic length $\lc$ in direction
$x$.
D)~Permeability normalized by the characteristic length $\lc$ in direction $z$.
\label{fig:lc}
}
\end{figure*}
%

Kozeny-Carman permeability estimation
%
\begin{equation}
\kappa = \frac{4 \Phi^{2}\,r^{2}}{150}\frac{\phi^{3}}{(1-\phi)^{2}}
\end{equation}
%

\begin{figure}
\includegraphics*[width=1.03\columnwidth]{data-figs/kozeny_carman}
\caption{
\label{fig:lc}
Kozeny-Carman permeabily estimation
}
\end{figure}

%%%%%%%%%%%%%%%%%%%%%%%%%%%%%%%%%%%%%%%%%%%%%%%%%%%%%%%
\subsection{Characteristic length}
\label{sec:lc}
In order to check the applicability of Katz-Thompson relation to our samples,
we calculate the characteristic length $\lc$. It is defined and calculated as
follows. At all points in the pore we define the largest sphere which can be
inscribed completely within the pore. Now consider building a continuous
pathways through the pore by joining these sphere, always taking the largest
one. The radius of the last sphere which completes a percolating pathway
through the sample is defined as $\lc$

Fig.~\ref{fig:lc} shows the results for $\lc$ in $x$- and $z$-directions.  As
it can be seen they show no significant difference (upper plots). Two lower
plots show the ratio of permeability to $\lc$ in the corresponding direction.
While Katz-Thompson seem to be qualitatively valid for $x$-direction, it can
be seen clearly that it fails in $z$-direction.

This leaves us with the question what is the exact mechanism behind such a
dramatic difference in two direction.

\subsection{Deposition of ellipsoids as reconstruction models for sandstones}
\label{sec:sandstone}
Here to present the results following Bibhu's idea.

%%%%%%%%%%%%%%%%%%%%%%%%%%%%%%%%%%%%%%%%%%%%%%%%%%%%%%%%%%%
\section{Conclusions}
\label{sec:conclusions}
To be written.

%%%%%%%%%%%%%%%%%%%%%%%%%%%%%%%%%%%%%%%%%%%%%%%%%%%%%%%%%%%%
\section*{Acknowledgements}
To be written.

%%%%%%%%%%%%%%%%%%%%%%%%%%%%%%%%%%%%%%%%%%%%%%%%%%%%%%%%%%%%%%%%%%%%%%%%%%%%
\bibliographystyle{bib/abbrv-unsrt}
\bibliography{bib/new_bib.bib}

% \begin{thebibliography}{99}
% \bibitem{ref:darcy} H. Darcy, Les Fontaines Publiques de la Ville de Dijon,
%   Dalmont, Paris (1985);

% \bibitem{ref:Scheidegger} A.E. Scheidegger, ``The Physics of flow through
%   porous media'', Univ. of Toronto Press, Toronto, (1974)

% \bibitem{ref:reza-pedro2011} Reza M. Baram, Pedro G. Lind, submitted to PRE
%   (2011).
% \end{thebibliography}
\end{document}
