\documentclass[aps,twocolumn,superscriptaddress,showpacs,showkeys]{revtex4}
%\documentclass[aps,superscriptaddress,showpacs]{revtex4}
\usepackage{graphics,graphicx,dcolumn,bm,fleqn,epic,eepic,float,epsfig}
\usepackage{amssymb,amsmath,multirow,rotate,color,float}
\usepackage{epstopdf}
\usepackage{times}
\usepackage{color}
\usepackage{soul}                    % package needed for overstriking
\definecolor{red}{rgb}{1,0,0}
\definecolor{green}{rgb}{0,1,0}
\definecolor{blue}{rgb}{0,0,1}
\newcommand{\xxr}[1]{\textcolor{red}{#1}}
\newcommand{\xxb}[1]{\textcolor{blue}{#1}}
%\usepackage{epsfig}
%\usepackage{graphicx}
\newcommand{\erf}{\text{erf}}

\begin{document}

\title{Transport properties of deposition of ellipsoidal particles}

\author{Bibhu Biswal}
\email{bbiswal@svc.ac.in}
\affiliation{New Delhi}

\author{Reza M.~Baram}
\email{reza@cii.fc.ul.pt}
\affiliation{Center for Theoretical and Computational Physics, 
             University of Lisbon,
             Av.~Prof.~Gama Pinto 2, 1649-003 Lisboa, Portugal}
\author{Jens Harting}
\email{j.harting@tue.nl}
\affiliation{TUE, Eindhoven}


\author{Pedro G.~Lind}
\email{plind@cii.fc.ul.pt}
\affiliation{Center for Theoretical and Computational Physics, 
             University of Lisbon,
             Av.~Prof.~Gama Pinto 2, 1649-003 Lisboa, Portugal}
\affiliation{Department of Physics, Faculty of Sciences of the
             University of Lisbon, 1649-003 Lisboa, Portugal} 

\author{Ariel Narvaez}
\email{a.e.narvaez.salazar@tue.nl}
\affiliation{TUE, Eindhoven}

\date{\today}

\begin{abstract}
We study the transport properties of depositions of ellipsoidal particles which has been generated using computer simulations.
To be written.

\end{abstract}

%%%%PACS e Keywords
\pacs{81.05.Rm,   %Granular materials, fabrication
           45.70.-n,   %Granular systems, classical mechanics of
           05.20.-y,   %Classical statistical mechanics
           45.70.Cc}  %Sandpile modelsk, Compaction granular systems

\keywords{Ellipsoidal, Deposition, Porous media, Permeability}

\maketitle

%%%%%%%%%%%%%%%%%%%%%%%%%%%%%%%%%%%%%%%%%%%%%%%%%%%%%%%%%%%%%%%%%%%%%%%%%%%%

\section{Introduction}
\label{sec:intro}

%Models for porous media can used both for reconstructing and fundamental study.

Predicting the transport properties of fluids in porous media based on their micro-structure 
is a challenging problem which has been the focus of many researches over last decades. 
For example, there have been many attempts to relate the permeability to 
a relevant length scale characterizing the geometrical structure of the porous medium. 
This makes sense since the permeability has units of length squared, as defined 
by Darcy's law [1,2] %\cite{ref:darcy  ref:scheidegger}:
\begin{equation}
q=\frac{-\kappa}{\mu} \nabla P,
\end{equation}
where $q$ is the flux (volumetric flow per unit cross-sectional area of the sample), 
$\nabla P$ is the pressure gradient across the sample and $\mu$ is the fluid viscosity.
There are numerous approaches based on some sort of length scales, such as hydraulic radius 
(the ratio of volume to surface of the porous material), which fail to give satisfactory 
predictions, except for simple cases.

Katz and Thompson \cite{ref:KatzThompson1986} reasoned that these attempts appeal to 
macroscopic geometrical properties and ignore the micro-structure of the porous material.
They derived, using percolation theory, a relation for the ratio of permeability $\kappa$ to electrical 
conductivity $\sigma$:
\begin{equation}
\frac{\kappa}{\sigma}=\frac{c l_c^2}{\sigma_0},
\label{eq:katz-thompson}
\end{equation}
where $\sigma_0$ is the electrical conductivity of the saturating fluid, $c$ is a constant of 
the order of $\frac{1}{226}$ and $l_c$ is a characteristic length. 
They estimated $l_c$ experimentally using mercury porosimetry measurements performed on rock samples. 
The results obtained for permeability using relation \ref{eq:katz-thompson} 
show impressive agreement with the experimental measurements. 

The characteristic length $l_c$ is obtained by finding the pathway 
which is both percolating and consists of largest spheres which be 
inscribed in the pore. 

In this study, we perform a computational study on the permeability of depositions of ellipsoid 
particles obtained in a recent work by two of the authors \cite{ref:reza-pedro2011}. 
We show that the permeability is strongly affected by the anisotropy of the samples [up to 5 times].
Furthermore, We will show that, while Katz-Thompson relation is valid when the permeability is 
measured in $x$-$y$ direction, it fails completely in $z$-direction. 

%%%%%%%%%%%%%%%%%%%%%%%%%%%%%%%%%%%%%%%%%%%%%%%%%%%%%%%%%%%%%%%%%%%%%%%%%%
\begin{figure}
\begin{center}
\includegraphics*[width=0.45\textwidth,angle=0]{fig1}
\caption{\protect
            (Color online)
            A deposit consisting of $N\sim 3000 $ ellipsoids 
            with shape parameters $\eta=2$ and $\zeta=1.5$.
            Colors are arbitrarily chosen for better visualization.}
\label{fig:deposit}
\end{center}
\end{figure}
%%%%%%%%%%%%%%%%%%%%%%%%%%%%%%%%%%%%%%%%%%%%%%%%%%%%%%%%%%%%%%%%%%%%%%%%%
 
The outline of the paper is as follows. 
In Sec.~\ref{sec:simulations} we describe preparation of the samples 
and the methods using which numerical measurements are performed.
In Sec.~\ref{sec:permeability} we present the results for permeability measurement. 
We examine the validity of Katz-Thompson relation for our samples in Sec.~\ref{sec:lc}.  
Discussion and conclusions are given in Sec.~\ref{sec:conclusions}.


%%%%%%%%%%%%%%%%%%%%%%%%%%%%%%%%%%%%%%%%%%%%%%%%%%%%%%%%%%%%%%%%%%%%%%%%
\section{Sample preparation and computational methods}
\label{sec:simulations}
For our study, we use deposits of ellipsoidal particles obtained using 
Molecular Dynamics simulations as described in \cite{ref:reza-pedro2011}.
The samples are generated by releasing such particles randomly inside a unit box, 
depositing in $-z$ direction under gravity. The box has periodic boundaries in $x$ and $y$ 
directions, limited from below by a wall and open from above. The particles are 
added to the system until the deposit attains a height slightly larger than one. 
In order to reduce the boundary effect, in all our analyses a layer of thickness $0.15$ 
from the bottom and $0.05$ from the top are removed.

The shape of an ellipsoid is characterized by two parameters
here defined as $\eta=a/b>1$ and $\zeta=b/c>1$, with $a \ge b \ge c$ being 
the semi-axis radii of the ellipsoid. 
Therefore, $\eta=1,\zeta>1$ corresponds to an oblate and $\eta>1,\zeta=1$ to a prolate, 
while  $\eta>1,\zeta>1$ is for a general ellipsoid. 

The size $r$ of an ellipsoid is defined as the radius of the sphere 
of equal volume. Therefore, the shape parameters $\eta \le 1$ and $\zeta \le 1$ 
together with the size $r>0$ fully specify an ellipsoid in a one-to-one manner.
 
In this work, all the samples under the consideration consist of identical particles.  
Therefore, we will refer to a sample by the parameters of the compositing particle. 
Figure \ref{fig:deposit} shows a deposit consisting of $N\sim 4200$ ellipsoids 
with $\eta=2$, $\zeta=1.5$ and $r=0.04$.


\begin{figure}
\begin{center}
\includegraphics*[width=0.48\textwidth,angle=0]{data-figs/porosity}
\caption{\protect
	Porosity of deposits as a function of the shape. $\eta$ and $\zeta$ 
	are the shape parameter of the compositing particles (see text.)
	}
\label{fig:porosity}
\end{center}
\end{figure}

To study the effect of the shape on the properties of the deposits, 
samples with the shape parameters from the range $1 \le \eta, \zeta \le 2.5$ 
are considered. It is known that the shape of the particles a strong effect on 
porosity of the deposits and, as it deviates from a sphere, leads to stronger 
anisotropy in the contact network and particle orientations \cite{ref:reza-pedro2011}. 
Figure \ref{fig:porosity} shows the porosity $\phi$ of deposits as a function of the shape. 

%%%%%%%%%%%%%%%%%%%%%%
\section{Permeability}
\label{sec:permeability}
\begin{figure}
\begin{center}
\includegraphics*[width=0.48\textwidth,angle=0]{data-figs/permeability_x}
\includegraphics*[width=0.48\textwidth,angle=0]{data-figs/permeability_z}
\caption{\protect
	Permeability as a function of the shape. $\eta$ and $\zeta$ 
	are the shape parameter of the compositing particles (see text.)
	}
\label{fig:perm}
\end{center}
\end{figure}

The calculation of permeability is done using Lattice Boltzmann simulation of a 
flow through the sample [Ariel may add some more details here]. 

The LB simulations are performed on the discretized version of the samples. Here, 
we use resolution $\frac{1}{256}$, that is, $256$ cubic cells per unit length in each direction. 
This resolution has been chosen such that it minimizes the computation time without loss of significant accuracy by 
comparing the results to that of resolutions $\frac{1}{512}$ and $\frac{1}{1024}$.

\begin{figure}
\begin{center}
\includegraphics*[width=0.48\textwidth,angle=0]{data-figs/permeability_ratio}
\caption{\protect
	Ratio of permeabilities in $z$- and $x$-directions as a function of the shape. 
	}
\label{fig:perm_ratio}
\end{center}
\end{figure}
As mentioned briefly in the previous section, the deposits are anisotropic in $z$-direction. 
Therefore, for all quantities under consideration we perform measurement in two directions, vertical 
horizontal. 

Figure \ref{fig:perm} shows the permeabilities measured in $x$- and $z$-direction as functions of the shape. 
One can see that the permeability in $z$-direction, $\kappa_z$, has generally a lower value $x$-direction, $\kappa_x$. This difference 
becomes more pronounced as the shape deviates from the sphere. This is better seen from the Fig. \ref{fig:perm_ratio} 
which shows the ratio $\kappa_z/\kappa_x$.

This is due to the fact that the particles tend to lie in the horizontal plane, 
increasing the surface area face the flow from vertical direction.
This is also the reason why flattening the particles (increasing $\zeta$) 
has strong effect on the permeability while elongation (increasing $\eta$) 
has no significant effect.

\begin{figure}[h]
\begin{center}
\includegraphics*[width=0.23\textwidth,angle=0]{data-figs/lc_x}
\includegraphics*[width=0.23\textwidth,angle=0]{data-figs/lc_z}
\includegraphics*[width=0.23\textwidth,angle=0]{data-figs/k_lc_x}
\includegraphics*[width=0.23\textwidth,angle=0]{data-figs/k_lc_z}
\caption{\protect
	Characteristic length $l_c$ in direction $x$ (upper left) and $z$ (upper right), and 
	the ratio of permeability to $l_c$ in the corresponding direction (two lower plots).
	}
\label{fig:lc}
\end{center}
\end{figure}

%%%%%%%%%%%%%%%%%%%%%%%%%%%%%%%%%%%%%%%%%%%%%%%%%%%%%%%
\subsection{Characteristic length}
\label{sec:lc}

In order to check the applicability of Katz-Thompson relation to our samples, 
we calculate the characteristic length $l_c$. It is defined and calculated as 
follows. At all points in the pore we define the largest sphere which can be inscribed 
completely within the pore. Now consider building a continuous pathways through the pore 
by joining these sphere, always taking the largest one. The radius of the last sphere 
which completes a percolating pathway through the sample is defined as $l_c$

Figure \ref{fig:lc} shows the results for $l_c$ in $x$- and $z$-directions. 
As it can be seen they show no significant difference (upper plots). Two lower plots 
show the ratio of permeability to $l_c$ in the corresponding direction.  While 
Katz-Thompson seem to be qualitatively valid for $x$-direction, it can be seen clearly that 
it fails in $z$-direction.

This leaves us with the question what is the exact mechanism behind such a dramatic difference in 
two direction.

\subsection{Deposition of ellipsoids as reconstruction models for sandstones}
\label{sec:sandstone}
Here to present the results following Bibhu's idea.

%%%%%%%%%%%%%%%%%%%%%%%%%%%%%%%%%%%%%%%%%%%%%%%%%%%%%%%%%%%%%%%%%%%%%%%%%%%%
\section{Conclusions}
\label{sec:conclusions}
To be written.
%%%%%%%%%%%%%%
\section*{Acknowledgements}
To be written.

%%%%%%%%%%%%%%%%%%%%%%%%%%%%%%%%%%%%%%%%%%%%%%%%%%%%%%%%%%%%%%%%%%%%%%%%%%%%
\begin{thebibliography}{99}
%%%%%%%%%%%%%%%%%%%%%%%%%%%%%%%%%%%%%%%%%%%%%%%%%%%%%%%%%%%%%%%%%%%%%%%%%%%%

\bibitem{ref:darcy} H. Darcy, Les Fontaines Publiques de la Ville de Dijon, Dalmont, Paris (1985);

\bibitem{ref:Scheidegger} A.E. Scheidegger, ``The Physics of flow through porous media'', 
	Univ. of Toronto Press, Toronto, (1974)

\bibitem{ref:reza-pedro2011} Reza M. Baram, Pedro G. Lind, submitted to PRE (2011).

\end{thebibliography}
\end{document}
%%%%%%%%%%%%%%%%%%%%%%%%%%%%%%%%%%%%%%%%%%%%%%%%%%%%%%%%%%%%%%%%%%%%%%%%%%%%
%%%%%%%%%%%%%%%%%%%%%%%%%%%%%%%%%%%%%%%%%%%%%%%%%%%%%%%%%%%%%%%%%%%%%%%%%%%%
